Analyzing movement patterns of people is a task that just recently became
feasible. With the help of GPS trackers the daily routine of people can
be analyzed \cite{geo1, geo2, geo3}, although the number of participants
is limited. A larger number of trips can for example be acquired
by installing GPS trackers in taxi cabs \cite{Ferreira2013, Guo2012}.
With the inception of bike sharing programs in various cities large
amounts of journey data from people daily utilizing bicycles have
been automatically created.

For companies providing shared bicycles it is interesting to see who
uses the bikes, when are they used, and which routes have been taken.
For maintenance reasons it is also important to know whether the usage
of the bikes are balanced between stations or whether there are some
stations that are used more often as start or destination.
Tourists or one time users may produce such imbalances, so it
would be beneficial to detect whether casual users follow any patterns.
Since bike sharing is popular among commuters, using it daily to get
to work, it is interesting to see how the usage patterns change during
the rush-hour. This information can be used to create infrastructure
to make commuting by bike easier.

In this paper we introduce a tool for analyzing temporal
origin-destination data and use it on data provided by
the bicycle sharing program
of Washington, DC: Capital Bikeshare \cite{wash}.
We will first present various approaches that have been used
to visualize origin-destination data (see Section~\ref{sec:rel}).
Then we will show the data set (Section~\ref{sec:data}) and
the implementation of the tool (Section~\ref{sec:impl}).
After that we present results (Section~\ref{sec:result}).
