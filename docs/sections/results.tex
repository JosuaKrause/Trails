\section{Results}
\label{sec:result}
By comparing trips during the morning rush-hour (time window from
6am to 9am) to trips during the day until the start of
the evening rush-hour (9am to 5pm) one can easily see
how different user groups use the bike sharing program.

\begin{figure*}
\centering
\includegraphics[width=3cm]{images/rush_7_tue.png}
\hspace*{0.5cm}
\includegraphics[width=3cm]{images/day_7_fri.png}
\hspace*{0.5cm}
\includegraphics[width=3cm]{images/day_7_sun.png}
\hspace*{0.5cm}
\includegraphics[width=3cm]{images/full_10_sun.png}
\hspace*{0.5cm}
\includegraphics[width=3cm]{images/full_10_wed.png}
\caption{Different time slices. The leftmost image shows
bike usage during the morning rush hour, the second shows
bike usage during the day on a week day. The third image
shows bike usage during the day on a Sunday.
The last two images show the bike usage of two complete
days (relevant window is 24 hours), the first being a Sunday
and the second being a weekday. All images are during the summer months.}
\label{fig:smallm}
\end{figure*}

The daily commuters during the rush hour have usually longer
journeys than users during the day. The commuters create
a temporary imbalance in the morning by bringing bicycles
to the center of the city where they work. However, the imbalance is
resolved in the evening during this rush hour. During the relatively
early time of the morning rush hour the number of casual users
is very low.

During the day the trips are much shorter and concentrate on the
center of the city. Casual users drive from the White House to
the Lincoln Memorial and vice versa implying this service is mainly
used by tourists visiting the city. The trips, however, are not
equally distributed in both directions which leads to slight (due to
the relatively small number) imbalance of bicycles in the touristy
area.

Comparing weekdays to weekends shows that during the weekend
the total number of trips far less than during the week.
Also the ratio of casual users is much higher.
This is likely due to day tourists visiting over the weekend
and the working commuters staying at home on the weekend.
However, trips on a weekend are not as much focused on the center
as trips during the mid day on the week.
This may come from
people that normally do not use the bike share but use it on
weekends to get to the city.

\begin{figure}[h]
\centering
\includegraphics[width=\linewidth]{images/barchart.png}
\caption{The bar chart view showing the number of trips per relevant
window.
In this bar chart the slice length is one day, the relevant window is
during the morning rush hour.
Weekends can be easily identified by a decreasing number of trips
which sometimes start already on fridays.
The current (leftmost) slice is a Saturday.}
\label{fig:bar}
\end{figure}

\section{Future Work}
The tool can be expanded to show multiple views of different
relevant windows at once to make comparison easier.
Also using data without fixed origins and destinations
can create new challenges of how to aggregate trips
effectively.
