\documentclass[journal]{vgtc}                % final (journal style)
%\documentclass[review,journal]{vgtc}         % review (journal style)
%\documentclass[widereview]{vgtc}             % wide-spaced review
%\documentclass[preprint,journal]{vgtc}       % preprint (journal style)
%\documentclass[electronic,journal]{vgtc}     % electronic version, journal

%% Uncomment one of the lines above depending on where your paper is
%% in the conference process. ``review'' and ``widereview'' are for review
%% submission, ``preprint'' is for pre-publication, and the final version
%% doesn't use a specific qualifier. Further, ``electronic'' includes
%% hyperreferences for more convenient online viewing.

%% Please use one of the ``review'' options in combination with the
%% assigned online id (see below) ONLY if your paper uses a double blind
%% review process. Some conferences, like IEEE Vis and InfoVis, have NOT
%% in the past.

%% Please note that the use of figures other than the optional teaser is not permitted on the first page
%% of the journal version.  Figures should begin on the second page and be
%% in CMYK or Grey scale format, otherwise, colour shifting may occur
%% during the printing process.  Papers submitted with figures other than the optional teaser on the
%% first page will be refused.

%% These three lines bring in essential packages: ``mathptmx'' for Type 1
%% typefaces, ``graphicx'' for inclusion of EPS figures. and ``times''
%% for proper handling of the times font family.

\usepackage{mathptmx}
\usepackage{graphicx}
\usepackage{times}

\newcommand{\etal}{\emph{et~al.~}}

%% We encourage the use of mathptmx for consistent usage of times font
%% throughout the proceedings. However, if you encounter conflicts
%% with other math-related packages, you may want to disable it.

%% This turns references into clickable hyperlinks.
\usepackage[bookmarks,backref=true,linkcolor=black]{hyperref} %,colorlinks
\hypersetup{
  pdfauthor = {Josua Krause},
  pdftitle = {Visualizing Bike Share Traffic},
  pdfsubject = {},
  pdfkeywords = {},
  colorlinks=true,
  linkcolor= black,
  citecolor= black,
  pageanchor=true,
  urlcolor = black,
  plainpages = false,
  linktocpage
}

%% If you are submitting a paper to a conference for review with a double
%% blind reviewing process, please replace the value ``0'' below with your
%% OnlineID. Otherwise, you may safely leave it at ``0''.
\onlineid{0}

%% declare the category of your paper, only shown in review mode
\vgtccategory{Research}

%% allow for this line if you want the electronic option to work properly
\vgtcinsertpkg

%% In preprint mode you may define your own headline.
%\preprinttext{To appear in an IEEE VGTC sponsored conference.}

%% Paper title.

\title{Visualizing Bike Share Traffic}

%% This is how authors are specified in the journal style

%% indicate IEEE Member or Student Member in form indicated below
\author{Josua Krause}
\authorfooter{
%% insert punctuation at end of each item
}

%other entries to be set up for journal
\shortauthortitle{Krause: Visualizing Bike Share Traffic}
%\shortauthortitle{Firstauthor \MakeLowercase{\textit{et al.}}: Paper Title}

%% Abstract section.
\abstract{
% TODO
} % end of abstract

%% Keywords that describe your work. Will show as 'Index Terms' in journal
%% please capitalize first letter and insert punctuation after last keyword
\keywords{
% TODO
}

%% ACM Computing Classification System (CCS). 
%% See <http://www.acm.org/class/1998/> for details.
%% The ``\CCScat'' command takes four arguments.

\CCScatlist{ % not used in journal version
 %\CCScat{K.6.1}{Management of Computing and Information Systems}%
%{Project and People Management}{Life Cycle};
 %\CCScat{K.7.m}{The Computing Profession}{Miscellaneous}{Ethics}
}

%% Uncomment below to include a teaser figure.
  \teaser{
  % TODO
 % \centering
%  \includegraphics[width=16cm]{CypressView}
 % \caption{In the Clouds: Vancouver from Cypress Mountain.}
  }

%% Uncomment below to disable the manuscript note
%\renewcommand{\manuscriptnotetxt}{}

%% Copyright space is enabled by default as required by guidelines.
%% It is disabled by the 'review' option or via the following command:
% \nocopyrightspace

%%%%%%%%%%%%%%%%%%%%%%%%%%%%%%%%%%%%%%%%%%%%%%%%%%%%%%%%%%%%%%%%
%%%%%%%%%%%%%%%%%%%%%% START OF THE PAPER %%%%%%%%%%%%%%%%%%%%%%
%%%%%%%%%%%%%%%%%%%%%%%%%%%%%%%%%%%%%%%%%%%%%%%%%%%%%%%%%%%%%%%%%

\begin{document}

%% The ``\maketitle'' command must be the first command after the
%% ``\begin{document}'' command. It prepares and prints the title block.

%% the only exception to this rule is the \firstsection command
\firstsection{Introduction}

\maketitle

%% \section{Introduction} %for journal use above \firstsection{..} instead

\documentclass[a4paper,twocolumn]{article}
\usepackage[utf8]{inputenc}
\usepackage{amsmath}
\usepackage{amsfonts}
\usepackage{amssymb}
\usepackage{graphicx}
\usepackage{microtype}
\usepackage{url}

\author{Josua Krause}

\begin{document}
\section*{Related Work}
Visualizing origin-destination data with temporal features
is a common task and multiple approaches have been used.

Guo~et~al.~\cite{Guo2006} use an origin-destination matrix
to show companies relocating within the US.
Rows and columns can be reordered to identify clusters
and patterns, and therefore provide no spatial context.
Also, the resolution of origins and destinations is limited
to states due to the use of a matrix.
In order to address those spatial problems
Wood~et~al.~\cite{Wood2002} explore nested matrices
that are laid over a geographic map.
Every cell of the matrix aggregates the origins
of the geographic map and contains
a second matrix showing aggregated destinations of
a smaller scale version of the map.
Becker~et~al.~\cite{Becker1995} compare
origin-destination matrices to node-link representations
and propose a dynamic node-link representation
where the user can define time intervals to aggregate the data,
set thresholds to limit the number of visible links,
and choose the regions that are displayed.

Using a node-link representation for showing origin-destination
data introduces clutter for large data-sets.
Holten and van Wijk~\cite{Holten2009} use edge bundling
to overcome clutter. However, this may imply hierarchical
relations of flows which do not exist.
Rae~\cite{Rae2009} use heat-maps based on
the density of migration flow of cities in the UK.
The user can also select a city to see its flow as node-links.
By computing density based clusters to define regions
Guo~et~al.~\cite{Guo2012} use choropleth maps to show
the net-flow of taxis in Hong Kong.
A kernel density representation shows the distribution
of destinations.
The same technique is used by Ferreira~et~al.~\cite{Ferreira2013}
for both destinations and origins.

Wood~et~al.~\cite{Wood2011}
Beecham~et~al.~\cite{Beecham2012}

TODO animation
TODO oscillation


\bibliographystyle{splncs}
\bibliography{trails}

\end{document}

%% if specified like this the section will be committed in review mode
%\acknowledgments{}

\bibliographystyle{abbrv}
%%use following if all content of bibtex file should be shown
%\nocite{*}
\bibliography{trails}
\end{document}
