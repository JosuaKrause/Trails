\documentclass[a4paper,twocolumn]{article}
\usepackage[utf8]{inputenc}
\usepackage{amsmath}
\usepackage{amsfonts}
\usepackage{amssymb}
\usepackage{graphicx}
\usepackage{microtype}
\usepackage{url}

\author{Josua Krause}

\begin{document}
\section*{Related Work}
Visualizing origin-destination data with temporal features
is a common task and multiple approaches have been used.

Guo~et~al.~\cite{Guo2006} use an origin-destination matrix
to show companies relocating within the US.
Rows and columns can be reordered to identify clusters
and patterns, and therefore provide no spatial context.
Also, the resolution of origins and destinations is limited
to states due to the use of a matrix.
In order to address those spatial problems
Wood~et~al.~\cite{Wood2002} explore nested matrices
that are laid over a geographic map.
Every cell of the matrix aggregates the origins
of the geographic map and contains
a second matrix showing aggregated destinations of
a smaller scale version of the map.
Becker~et~al.~\cite{Becker1995} compare
origin-destination matrices to node-link representations
and propose an animated dynamic node-link representation
where the user can define time intervals to aggregate the data,
set thresholds to limit the number of visible links,
and choose the regions that are displayed.
Tversky~et~al.~\cite{Tversky2002} states that
keeping track of changes is difficult when using
animation. However, animation can be used to
identify trends (Heer and Robertson~\cite{Heer2007},
and Robertson~et~al.~\cite{Robertson}) and
patterns (Griffin and MacEachren\cite{Griffin2006}).

Using a node-link representation for showing origin-destination
data introduces clutter for large data-sets.
Holten and van Wijk~\cite{Holten2009} use edge bundling
to overcome clutter. However, this may imply hierarchical
relations of flows which do not exist.
Rae~\cite{Rae2009} use heat-maps based on
the density of migration flow of cities in the UK.
The user can also select a city to see its flow as node-links.
By computing density based clusters to define regions
Guo~et~al.~\cite{Guo2012} use choropleth maps to show
the net-flow of taxis in Hong Kong.
A kernel density representation shows the distribution
of destinations.
The same technique is used by Ferreira~et~al.~\cite{Ferreira2013}
for both destinations and origins.

Boyandin~et~al.~\cite{Boyandin2011} displays origins and
destinations of migration flows on separate maps connected
by links to a time-table showing the magnitude of the flows
for a given time.
Another way to avoid clutter in node-link
flow representations is to aggregate flows with
similar origins and destinations.
Wood~et~al.~\cite{Wood2011} does this with bicyclist data
from London showing asymmetric bezier curves as links to avoid
over-plotting of opposing flows.
Beecham~et~al.~\cite{Beecham2012} improves
this by also showing hourly densities as cycle
plot allowing selections on the input data by brushing.

\bibliographystyle{splncs}
\bibliography{trails}

\end{document}