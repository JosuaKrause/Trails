\documentclass[a4paper,twocolumn]{article}
\usepackage[utf8]{inputenc}
\usepackage{amsmath}
\usepackage{amsfonts}
\usepackage{amssymb}
\usepackage{graphicx}
\usepackage{microtype}
\usepackage{url}

\author{Josua Krause}

\begin{document}
\section*{Related Work}
Visualizing origin-destination data with temporal features
is a common task and multiple approaches have been used.

Guo~et~al.~\cite{Guo2006} use an origin-destination matrix
to show companies relocating within the US.
Rows and columns can be reordered to identify clusters
and patterns, and therefore provide no spatial context.
Also, the resolution of origins and destinations is limited
to states due to the use of a matrix.
In order to address those spatial problems
Wood~et~al.~\cite{Wood2002} explore nested matrices
that are laid over a geographic map.
Every cell of the matrix aggregates the origins
of the geographic map and contains
a second matrix showing aggregated destinations of
a smaller scale version of the map.
Becker~et~al.~\cite{Becker1995} compare
origin-destination matrices to node-link representations
and propose a dynamic node-link representation
where the user can define time intervals to aggregate the data,
set thresholds to limit the number of visible links,
and choose the regions that are displayed.

Using a node-link representation for showing origin-destination
data introduces clutter for large data-sets.
Holten and van Wijk~\cite{Holten2009} use edge bundling
to overcome clutter. However, this may imply hierarchical
relations of flows which do not exist.

Rae~\cite{Rae2009} use flow density based heat-maps




\bibliographystyle{splncs}
\bibliography{trails}

\end{document}